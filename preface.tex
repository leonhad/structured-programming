\chapter*{PREFACE}
\addcontentsline{toc}{chapter}{\protect\numberline{}PREFACE}%

\noindent
In recent years there has been an increasing interest in the art of computer programming, the conceptual tools available for the design of programs, and the prevention of programming oversights and error. The initial out- standing contribution to our understanding of this subject was made by E. W. Dijkstra, whose Notes on Structured Programming form the first and major section of this book. They clearly expound the reflections of a brilliant programmer on the methods which he has hitherto unconsciously applied; there can be no programmer of the present day who could not increase his skills by a study and conscious application of these principles.

In the second monograph I have tried to describe how similar principles can be applied in the design of data structures. I have suggested that in analysing a problem and groping towards a solution, a programmer should take advantage of abstract concepts such as sets, sequences, and mappings; and judiciously postpone decisions on representation until he is constructing the more detailed code of the program. The monograph also describes a range of useful ideas for data representation, and suggests the criteria relevant for their selection.

The third monograph provides a synthesis of the previous two, and expounds the close theoretical and practical connections between the design of data and the design of programs. It introduces useful additional methods for program and data structuring which may be unfamiliar to many programmers. The examples show that structured programming principles can be equally applied in ``bottom-up'' as in ``top-down'' program design. The original inspiration, insight, and all the examples were contributed by O.-J. Dahl; I have only assembled the material, and added some additional explanations where I found it difficult to understand.
\bigskip
 
\noindent
June 1972\hfill C. A. R. HOARE
