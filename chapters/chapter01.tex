\chapter{Notes on Structured Programming}

EDSGER W. DIJKSTRA

\section{To my reader}

These notes have the status of ``Letters written to myself'': I wrote them down because, without doing so, I found myself repeating the same arguments over and over again. When reading what I had written, I was not always too satisfied.

For one thing, I felt that they suffered from a marked verbosity. Yet I do not try to condense them (now), firstly because that would introduce another delay and I would like to ``think on'', secondly because earlier experiences have made me afraid of being misunderstood: many a programmer tends to see his (sometimes rather specific) difficulties as the core of the subject and as a result there are widely divergent opinions as to what programming is really about.

For another thing, as a document this is very incomplete: I am only too aware of the fact that it ends in mid-air. Yet I have decided to have these notes duplicated, besides some practical considerations mainly to show what I have thought to those who expressed interest in it or to those whose comments I would welcome.

I hope that, despite its defects, you will enjoy at least parts of it. If these notes prove to be a source of inspiration or to give you a new appreciation of the programmer’s trade, some of my goals wil1 have been reached.
\bigskip

\hfill Edsger W. Dijkstra	


\section{On our inability to do much}

I am faced with a basic problem of presentation. What I am really concerned about is the composition of large programs, the text of which may be, say, of the same size as the whole text of this booklet. Also I have to include examples to illustrate the various techniques. For practical reasons, the demonstration programs must be small, many times smaller than the ``life-size programs'' I have in mind. My basic problem is that precisely this difference in scale is one of the major sources of our difficulties in programming!

It would be very nice if I could illustrate the various techniques with small demonstration programs and could conclude with ``...and when faced with a program a thousand times as large, you compose it in the same way.'' This common educational device, however, would be self-defeating as one of my central themes will be that any two things that differ in some respect by a factor of already a hundred or more, are utterly incomparable.

History has shown that this truth is very hard to believe. Apparently we are too much trained to disregard differences in scale, to treat them as ``gradual differences that are not essential''. We tell ourselves that what we can do once, we can also do twice and by induction we fool ourselves into believing that we can do it as many times as needed, but this is just not true! A factor of a thousand is already far beyond our powers of imagination!

Let me give you two examples to rub this in. A one-year old child will crawl an all fours with a speed of, say, one mile per hour. But a speed of a thousand miles per hour is that of a supersonic jet. Considered as objects with moving ability the child and the jet are incomparable, for whatever one can do the other cannot and vice versa. Also: one can close one’s eyes and imagine how it feels to be standing in an open place, a prairie or a sea shore, while far away a big, reinless horse is approaching at a gallop, one can ``see'' it approaching and passing. To do the same with a phalanx of a thousand of these big beasts is mentally impossible: your heart would miss a number of beats by pure panic, if you could!

To complicate matters still further, problems of size do not only cause me problems of presentation, but they lie at the heart of the subject: widespread underestimation of the specific difficulties of size seems one of the major underlying causes of the current software failure. To all this I can see only one answer, viz. to treat problems of size as explicitly as possible. Hence the title of this section.

To start with, we have the “size” of the computation, i.e. the amount of information and the number of operations involved in it. It is essential that this size is large, for if it were really small, it would be easier not to use the computer at all and to do it by hand. The automatic computer owes it right to exist, its usefulness, precisely to its ability to perform large computations where we humans cannot. We want the computer to do what we could never do ourselves and the power of present-day machinery is such that even small computations are by their very size already far beyond the powers of our unaided imagination.

Yet we must organize the computations in such a way that our limited powers are sufficient to guarantee that the computation will establish the desired effect. This organizing includes the composition of the program and here we are faced with the next problem of size, viz. the length of the program text, and we should give this problem also explicit recognition. We should remain aware of the fact that the extent to which we can read or write a text is very much dependent on its size. In my country the entries in the telephone directory are grouped by town or village and within each such group the subscribers are listed by name in alphabetical order. I myself live in a small village and given a telephone number I have only to scan a few columns to find out to whom the telephone number belongs, but to do the same in a large city would be a major data processing task!

It is in the same mood that I should like to draw the reader's attention to the fact that ``clarity'' has pronounced quantitative aspects, a fact many mathematicians, curiously enough, seem to be unaware of. A theorem stating the validity of a conclusion when ten pages full of conditions are satisfied is hardly a convenient tool, as all conditions have to be verified whenever the theorem is appealed to. In Euclidean geometry, Pythagoras’ Theorem holds for any three points A, B and C such that through A and C a straight line can be drawn orthogonal to a straight line through B and C. How many mathematicians appreciate that the theorem remains applicable when some or all of the points A, B and C coincide? Yet this seems largely responsible for the convenience with which Pythagoras Theorem can be used.

Summarizing: as a slow-witted human being I have a very small head and I had better learn to live with it and to respect my limitations and give them full credit, rather than to try to ignore them, for the latter vain effort will be punished by failure.
