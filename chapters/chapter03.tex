\chapter{Hierarchical Program Structures}

{
	\noindent
	\scshape\hfill\scriptsize Ole-Johan Dahl and C. A. R. Hoare\hfill
}
\renewcommand{\leftmark}{\normalfont\scriptsize\hfill OLE-JOHAN DAHL AND C. A. R. HOARE\hfill}

\section{Introduction}

In this monograph we shall explore certain ways of program structuring and point out their relationship to concept modeling.

We shall make use of the programming language SIMULA 67 with particular emphasis on structuring mechanisms. SIMULA 67 is based on ALGOL 60 and contains a slightly restricted and modified version of ALGOL 60 as a subset. Additional language features are motivated and explained informally when introduced. The student should have a good knowledge of ALGOL 60 and preferably be acquainted with list processing techniques.

For a full exposition of the SIMULA language we refer to the ``Simula 67 Common Base Language'' [\hyperref[ref:2]{2}]. Some of the linguistic mechanisms introduced in the monograph are currently outside the ``Common Base''\footnote{The Simula 67 language was originally designed at the Norwegian Computing Center, Oslo. The Common Base defines those language features which are common to all implementations. The Common Base is continually being maintained and revised by the ``Simula Standards Group'', each of whose members represents an organization responsible for an implementation. 8 organizations are currently represented on the SSG. (Summer 1971).}. The monograph is an extension and reworking of a series of lectures given by Dahl at the NA TO Summer School on Programming, Marktoberdorf 1970. Some of the added material is based on programming examples that have occurred elsewhere [\hyperref[ref:3]{3}, \hyperref[ref:4]{4}, \hyperref[ref:5]{5}].

\section{Preliminaries}

Our subject matter as programmers is a special class of dynamic system, which we call computing processes or data processes. A programming language provides us with basic concepts and composition rules for constructing and analyzing computing processes.

The following are some of the basic concepts provided by ALGOL 60.

Page 176






\bigskip

\noindent
\textbf{REFERENCES}
\addcontentsline{toc}{section}{References}
\medskip\nopagebreak

\begin{enumerate}[leftmargin=*, itemsep=.1em, wide=0pt, align=left, label=(\arabic*)]
	\item \label{ref:1}
	Naur, P. (ed.) (1962/63). Revised Report on the Algorithmic Language. ALGOL 60. \textit{Comp. J.}, \textbf{5}, pp. 349--367.
	
	\item \label{ref:2}
	Dahl, 0.-J., Myhrhaug, B., Nygaard, K. (1968). The Simular 67 Common Base Language. Norwegian Computing Centre, Forskningsveien 1B, Oslo 3.
	
	\item \label{ref:3}
	Wang, A., Dahl, 0.-J. (1971). Coroutine Sequencing in a Block Structured Environment. \textit{BIT} \textbf{11}, 4, pp. 425--449.
	
	\item \label{ref:4}
	Dahl, 0.-J., Nygaard (1966). Simula --- an Algol-Based Simulation Language. \textit{Comm. A.C.M.} \textbf{9}, 9, pp. 671--678. 
	
	\item \label{ref:5}
	Dahl, 0.-J. (1968). Discrete Event Simulation Languages. ``Programming Languages'' (ed. Genuys, F.). pp. 349--395. Academic Press, London.
	
	\item \label{ref:6}
	Hoare, C. A. R. (1968). Record Handling. ``Programming Languages'' (ed. Genuys, F.). pp. 291--347. Academic Press, London.
	
	\item \label{ref:7}
	Conway, M. E. (1963). Design of a Separable Transition --- Diagram Compiler. \textit{Comm. A.C.M.} 6, 7, pp. 396--408.
	
	\item \label{ref:8}
	Naur, P. (1969). Programming by Actions Clusters. \textit{BIT} \textbf{9}, 3, pp. 250--258.
	
	\item \label{ref:9}
	Dijkstra, E. W. (1972). Notes on Structured Programming. ``Structured Programming''. pp. 1--82. Academic Press, London.
	
	\item \label{ref:10}
	Knuth, D. E., McNeley, J. L. (1964). SOL --- A Symbolic Language for General-Purpose Systems Simulation. IEEE Trans. E.C.
	
	\item \label{ref:11}
	IBM, General Purpose Systems Simulator.
	
	\item \label{ref:12}
	Dijkstra, E. W. (1968). Co-operating Sequential Processes. ``Programming Languages''. pp. 43--112. Academic Press, London.
\end{enumerate}